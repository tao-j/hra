\PassOptionsToPackage{unicode=true}{hyperref} % options for packages loaded elsewhere
\PassOptionsToPackage{hyphens}{url}
%
\documentclass[]{article}
\usepackage{lmodern}
\usepackage{amssymb,amsmath}
\usepackage{ifxetex,ifluatex}
\usepackage{fixltx2e} % provides \textsubscript
\ifnum 0\ifxetex 1\fi\ifluatex 1\fi=0 % if pdftex
  \usepackage[T1]{fontenc}
  \usepackage[utf8]{inputenc}
  \usepackage{textcomp} % provides euro and other symbols
\else % if luatex or xelatex
  \usepackage{unicode-math}
  \defaultfontfeatures{Ligatures=TeX,Scale=MatchLowercase}
\fi
% use upquote if available, for straight quotes in verbatim environments
\IfFileExists{upquote.sty}{\usepackage{upquote}}{}
% use microtype if available
\IfFileExists{microtype.sty}{%
\usepackage[]{microtype}
\UseMicrotypeSet[protrusion]{basicmath} % disable protrusion for tt fonts
}{}
\IfFileExists{parskip.sty}{%
\usepackage{parskip}
}{% else
\setlength{\parindent}{0pt}
\setlength{\parskip}{6pt plus 2pt minus 1pt}
}
\usepackage{hyperref}
\hypersetup{
            pdftitle={Rank Aggregation},
            pdfauthor={Tao Jin},
            pdfborder={0 0 0},
            breaklinks=true}
\urlstyle{same}  % don't use monospace font for urls
\setlength{\emergencystretch}{3em}  % prevent overfull lines
\providecommand{\tightlist}{%
  \setlength{\itemsep}{0pt}\setlength{\parskip}{0pt}}
\setcounter{secnumdepth}{0}
% Redefines (sub)paragraphs to behave more like sections
\ifx\paragraph\undefined\else
\let\oldparagraph\paragraph
\renewcommand{\paragraph}[1]{\oldparagraph{#1}\mbox{}}
\fi
\ifx\subparagraph\undefined\else
\let\oldsubparagraph\subparagraph
\renewcommand{\subparagraph}[1]{\oldsubparagraph{#1}\mbox{}}
\fi

% set default figure placement to htbp
\makeatletter
\def\fps@figure{htbp}
\makeatother


\title{Rank Aggregation}
\author{Tao Jin}
\date{}

\begin{document}
\maketitle

\hypertarget{introduction-and-motivation}{%
\section{Introduction and
Motivation}\label{introduction-and-motivation}}

In rank aggregation tasks, specifically in the case of given pairwise
comparison to recover the whole series of rank, order obtained from one
pair may be contradict to the other. It is natural to incorporate this
kind of noisyness into aggregation model so as to reach a consensus the
populartion would agree to.

While considering the trustworthyness of all the comparison pairs, it is
also notable that the annotator/judge will have various error rate while
making the judgements. It may be beneficial to also take the quality of
the annoator into account. For instances, in some area, only a small
portion of the experts can make sound judgment most of the time. And in
some problem regarding to common sense, majority of the annotators may
give a reasonable mount of reliable statment. In many rank aggregations,
although model consider the effect of noisy inputs by inprecise
judgement claimed, there is no account for the difference of amount of
noisyness provided by each judge mentioned above.

In order to model the noisyness with such granurity, we introduce a
natural extention to Bradley-Terry model (BTL) which provides
flexibility to handle this situation. We call this algorithm generalized
BTL model (GBTL).

In this paper, we first introduce the standard BTL model and then show
the effectiveness of our GBTL model on sythetic data as well as Reading
Level Dataset.

\hypertarget{related-work}{%
\section{Related Work}\label{related-work}}

Spetral MLE

Crowd BT

\hypertarget{gbtl-algorithm}{%
\section{GBTL: Algorithm}\label{gbtl-algorithm}}

\hypertarget{problem-setup}{%
\subsection{Problem Setup}\label{problem-setup}}

Given a set of items and each of them have a conceptual score which can
be used to rank them in a certain order, which may or may not exist in
reality. The score (sometimes called the utility of item) is denoted as
\(\mathbf{s}\). In some literature it is referred as \(\mathbf{w}\) or
\(\pi\). In this document, we refer \(w\) as \(w=e^s\) and assume
\(\sum{\pi_i} = 1\), which is a normalized version of \(w\).

Suppose there are a group of \(m\) judges who are presented with pair of
objects to make judgment. And there are \(n\) items to be compared. Let
\(c_{ji,k} (i, j \in [n], k \in[m])\) be the numbers of times \(i\) is
preferred over item \(j\) reported by judge \(k\). Note that
\(i \neq j\).

In following methods, each kind of likelihood function corredponds to a
model with various intentions.

\hypertarget{simple-bradley-terry-model}{%
\subsection{Simple Bradley Terry
Model}\label{simple-bradley-terry-model}}

We assume each user have a random noise
\(\epsilon \sim Gumble(\mu, \beta)\) added to the conceptual score
\(s_i\) while looking at item \(i\). So the probability of \(s_i > s_j\)
is actually \(\Pr(s_i + \epsilon_i > s_j + \epsilon_j)\). Here, the BTL
model assumes \(\beta = 1\).

We can use integral or moment to prove the probablity for event item
\(i\) ranked before item \(j\) to happen is as follows:

\[Pr(I_i > I_j) = \frac{e^{s_i}}{e^{s_i} + e^{s_j}} = \left( {e^{s_j - s_i} + 1} \right)^{-1} \]

After collecting result from judges, we can have a counting matrix
\(C\), which stores number of times the respective statement appeared.
Since the number of comparisons can vary case by case, the actual
likelihood is normalized by of total number of comparisons. Then we can
write the log likelihood of the dataset as

\[L_{BTL}(\mathbf{s}) = -(\sum{c_{ij}})^{-1} \sum{\left( c_{ij} \cdot \ln \left( {e^{s_j - s_i} + 1} \right) \right) }  \]

\hypertarget{crowd-bt-model}{%
\subsection{Crowd BT Model}\label{crowd-bt-model}}

In order to account for user quiality, author simplly added a quality
parameter \(\eta\) for each judge.

\[Pr(I_i > I_j) = \eta_k \cdot \frac{e^{s_i}}{e^{s_i} + e^{s_j}} + ( 1- \eta_k ) \cdot \frac{e^{s_j}}{e^{s_i} + e^{s_j}} 
= \eta_k \left( {e^{s_j - s_i} + 1} \right)^{-1} + (1 -\eta_k) \left( {e^{s_i - s_j} + 1} \right)^{-1} \]

\hypertarget{generalized-bt-model}{%
\subsection{Generalized BT Model}\label{generalized-bt-model}}

Apart from directly plug in a quality factor for each user. To account
for trustworyhness of a user, we can also do this by adjusting the
\(\beta\) parameter noise generation Gumble distribution.

\[-(\sum{c_{ij, k}})^{-1} \sum{c_{ij,k} \cdot \ln({e^{\frac{s_j - s_i}{\beta_k}} + 1}})\]

By definition, the Gumble distribution must have \(\beta > 0\) . In
order to cast the restriction that \(\beta > 0\), in actual
implementation \(\beta = \epsilon^2\) is used, the formula becomes
\(-(\sum{c_{ij, k}})^{-1} \sum{c_{ij,k} \cdot \ln({e^{\frac{s_j - s_i}{\epsilon_k^2}} + 1}})\).

\hypertarget{relaxed-generalized-bt-model}{%
\subsection{Relaxed Generalized BT
Model}\label{relaxed-generalized-bt-model}}

Actually the restrcition \(\beta > 0\) can be removed in \texttt{gbtl}.

Suppose a judge have good knowlege of the items, however he tends to
provide ratings at the opposite direction to his best knowledge. In such
adversarial setting, the fomula is quite similar to previous case.
Assume \(\beta'_k = - \beta_k < 0\)

\[-(\sum{c_{ij, k}})^{-1} \sum{c_{ij,k} \cdot \ln({e^{\frac{s_j - s_i}{-\beta_k}} + 1}})\]
we can see that it is still calculating the probability of items i
better than item j.

Ideally, this algorithm can handle those adversarial judges which know
that \(s_i > s_j\) but cast a vote that \(s_i < s_j\).

\[-(\sum{c_{ij, k}})^{-1} \sum{c_{ij,k} \cdot \ln({e^{\frac{s_j - s_i}{\beta'_k}} + 1}})\]

When \(\beta\) is near zero, it will correspond to a perfect judge. When
noisy comparison is seen, in order to compensate the likelihood, the
\(\beta\) tends to be larger and larger. It would be very hard for the
optimization algorithm which is a likelihood maximizer to produce
gradient that to make the value of \(\beta\) to become negative. For
this knid of judge, when \(\beta\) is positive and very small, it will
have smaller likelihood than \(\beta\) is positive and have larger
value. However, when \(\beta\) is negative and very small, it will also
have larger likelihood. To overcome this next algorithm is used.

\hypertarget{gbtl-inverse}{%
\subsubsection{gbtl-inverse}\label{gbtl-inverse}}

\texttt{gbtl-inverse}: Let \(\gamma_k = 1/\beta_k\) to replace the
denominator part in \texttt{gbtl-negative} as a multiplier for easy
optimization.

This is used to compensate the effect of previous method, so the good
judge will have a very large \(\gamma_k\), while bad judge can flip sign
to become adversarial while the gradient will always in the right
direction to make the likelihood larger.

\[-(\sum{c_{ij, k}})^{-1} \sum{c_{ij,k} \cdot \ln({e^{(s_j - s_i) \cdot \gamma_k} + 1}})\]

\hypertarget{initialization}{%
\subsection{Initialization}\label{initialization}}

Instead of put initial value of each parameter randomly, the spectral
method described in ``Rank Centrality'' is used to provide a near
estimate of the true utility, so that the gradient descent algorithm may
produce better result.

\hypertarget{optimization}{%
\subsection{Optimization}\label{optimization}}

\begin{itemize}
\tightlist
\item
  Calculate the gradient using likelihood function mentioned above.
  Update \(s\) and \(\beta\) simultaneously. It is also possible to do
  alternating update.
\end{itemize}

Because in the likelihood function, only difference between \(s\)
matters, so there will be infinate number of solutions if \(s\) is not
restricted. Without loss of generality we put \(\min(s) = 0\). In
\texttt{gbtl-*} model, consider we have fixed \texttt{s}, but the +
\(s\) is shifted to have minimum value to be 0.
\(\mathbf{s} = \mathbf{s} - \min{s_i}\). + \(\beta\) is scaled to
prevent the change of likelihood for next step
\(\beta = \beta / \sum{s_i}\). + \(s\) is then nomalized/scaled to be
summed to 1. \(\mathbf{s} = \mathbf{s} / \sum{s_i}\).

\hypertarget{experiments}{%
\section{Experiments}\label{experiments}}

\hypertarget{synthetic-data}{%
\subsection{Synthetic data}\label{synthetic-data}}

\hypertarget{reading-level-data}{%
\subsection{Reading Level data}\label{reading-level-data}}

\end{document}
